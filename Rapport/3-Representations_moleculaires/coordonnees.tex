\par La matrice des coordonnées atomiques est la façon la plus simple de représenter la géométrie d'une molécule. L'intérêt de cette représentation est qu'elle est utilisée par les chimistes (fichiers .mol, .xyz + utilisation dans les logiciels de calcul ?). Il s'agit donc pour nous d'une représentation d'entrée et de sortie. Nos données d'apprentissage contiennent pour chaque molécule une matrice des positions, en plus des numéros et masses atomiques, et nous devons être capables de fournir cette représentation en sortie de nos prédictions, pour que nos résultats soient utilisables par les chimistes.\\

\par Formellement, la matrice des coordonnées atomiques d'une molécule contient les coordonnées de chaque atome dans un repère cartésien orthonormé à trois dimensions.

\begin{figure}[!h]
	\centering
	
	\begin{tabular}{|c|c|c|c|}
		\hline
		$\boldsymbol{x_1}$ & $\boldsymbol{y_1}$ & $\boldsymbol{z_1}$ \\ \hline	
		$\boldsymbol{x_2}$ & $\boldsymbol{y_2}$ & $\boldsymbol{z_2}$ \\ \hline	
		\textbf{\rot{... }} & \textbf{\rot{... }} & \textbf{\rot{... }}\\ \hline 	
		$\boldsymbol{x_n}$ & $\boldsymbol{y_n}$ & $\boldsymbol{z_n}$ \\ \hline	
	\end{tabular}

	\caption{Matrice des coordonnées atomiques (molécule de taille $n$)}
\end{figure}