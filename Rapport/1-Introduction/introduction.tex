\addcontentsline{toc}{chapter}{Introduction}  

\par Dans le cadre de ma première année de Master en Informatique à l'université d'Angers, j'ai effectué un Travail Encadré de Recherche (TER) au sein du département informatique de la faculté des Sciences d'Angers, et plus précisément au sein du projet QuChemPedia (\ref{quchempedia}). Ce travail, d’une durée de 10 semaines, a consisté à utiliser des modèles d'apprentissage automatique afin d'effectuer des prédictions en chimie quantique.\\

\par Plus précisément, les objectifs initiaux comprenaient la conception et l'implémentation de réseaux de neurones artificiels, l'entraînement de modèles prédictifs sur des données réelles de chimie quantique, ainsi la comparaison des performances relatives des modèles. Au cours de la réalisation du projet, d'autres objectifs ont émergé. Un travail d'analyse des données moléculaires a notamment permis d'optimiser les performances des modèles. De plus, une volonté de mise en perspective des résultats obtenus a mené à l'entraînement d'autres types de modèles.\\

\par Mon travail s'est scindé en deux parties distinctes. La première partie (chapitre \ref{delta_dist_chap}) avait pour but de reproduire des résultats antérieurs, d'établir de nouvelles façons de représenter la géométrie des molécules (\ref{repr_mat_pts_fixes}), et éventuellement d'améliorer les performances des modèles déjà existants. Les modèles suivant l'approche initiale se sont cependant révélés peu efficaces, c'est pourquoi la mise en place d'une nouvelle approche (chapitre \ref{dist_rel_chap}) a constitué la seconde partie de mon travail.\\

\par La nature expérimentale du travail que j'ai effectué a rendu difficile la planification des différentes tâches. J'ai néanmoins établi a posteriori un diagramme de Gantt représentant la durée et la chronologie des grandes parties de ce travail, visible en annexe \ref{gantt}.
