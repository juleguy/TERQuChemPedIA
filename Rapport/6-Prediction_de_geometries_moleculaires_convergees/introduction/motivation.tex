\par L'objectif des modèles prédictifs que l'on décrit dans ce chapitre est de prédire la géométrie convergée (REF GEOM CONVERG) d'une molécule complète, à partir d'une géométrie non convergée. Ils sont issus d'une tentative de reproduction de résultats antérieurs, afin de confirmer la méthode élaborée lors des stages précédents sur le projet QuChemPedIA.\\
Chronologiquement, ces modèles ont constitué la première partie de mon travail, avant de passer aux modèles tentant de prédire les longueurs de liaisons (REF DIST\_REL), à cause de l'impossibilité de produire des prédictions de qualité suffisante (REF RESULTATS).\\

\par L'objectif à terme de ces modèles est de pouvoir constituer une alternative au DFT (REF DFT) pour calculer rapidement la géométrie convergée d'une molécule. Cela nécessite de produire des prédictions d'une très grande précision. Cependant, le but ici est avant tout de valider une méthode et notre capacité à produire des prédictions d'ordre géométrique. Nous ne cherchons donc pas à créer un modèle effectuant de très bonnes prédictions, mais plutôt à définir une représentation des données et un ensemble de paramètres permettant d'obtenir de bons résultats.